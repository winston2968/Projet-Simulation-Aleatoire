

\usepackage[top=3cm, bottom=3cm, left=3.5cm, right=3.5cm]{geometry}


\setlength{\parindent}{0cm}


\usepackage[french]{babel}
\usepackage[T1]{fontenc}
\usepackage{ragged2e}
\usepackage{amsfonts}
\usepackage{systeme}
\usepackage{amsmath}
\usepackage{amssymb}
\usepackage{comment}
\usepackage{multicol}
\usepackage{lipsum} 
\usepackage{graphicx}
\usepackage{stmaryrd}
\usepackage{systeme}
\usepackage{wrapfig}
\usepackage{colortbl}
\usepackage{cellspace}
\usepackage{stmaryrd}
\usepackage{ntheorem}
\usepackage{lmodern}
\usepackage{mathtools}
\usepackage{ragged2e}
\usepackage{tabularx}
\usepackage{titlepic}
\usepackage{fancyhdr}
\usepackage{caption}
\usepackage{xcolor} % pour les couleurs
\usepackage[linkbordercolor=white]{hyperref} % après avoir chargé xcolor
\usepackage{systeme}
\usepackage[T1]{fontenc}
\usepackage{lmodern}
\usepackage{listings}
\usepackage{tikz}
\usepackage{mdframed}
\usepackage{xparse} % Nécessaire pour définir des environnements avec arguments optionnels
\usepackage{booktabs}  % Pour des tableaux plus jolis
\usepackage{float} % For image positionning


% \setcounter{secnumdepth}{-1} % Désactive le compteur des parties

% PAGE SETTINGS

% \justifying

% \setlength{\columnseprule}{1pt}
% \def\columnseprulecolor{\color{black}}

% \newcolumntype{C}{>{$\displaystyle}Sc<$}
% \cellspacetoplimit=5pt
% \cellspacebottomlimit=5pt

\newcolumntype{C}{>{$\displaystyle}Sc<$}
\cellspacetoplimit=5pt
\cellspacebottomlimit=5pt


% MATHS SHORTHANDS

\newcommand{\C}{\mathbb{C}}
\newcommand{\R}{\mathbb{R}}
\newcommand{\Q}{\mathbb{Q}}
\newcommand{\Z}{\mathbb{Z}}
\newcommand{\N}{\mathbb{N}}
\newcommand{\U}{\mathbb{U}}
\newcommand{\K}{\mathbb{K}}
\newcommand{\M}{\mathcal{M}}
\newcommand{\B}{\mathcal{B}}

\renewcommand{\epsilon}{\varepsilon}
\renewcommand{\phi}{\varphi}
\renewcommand{\rho}{\varrho}

% MOD NOTATION

\theorembodyfont{\upshape}

% Définir un environnement pour encadrer les définitions avec un titre
\NewDocumentEnvironment{definition}{O{}}
{
  \begin{mdframed}[linewidth=0pt,linecolor=gray,backgroundcolor=gray!10,roundcorner=5pt]
  \textbf{Définition}%
  \IfNoValueTF{#1}{}{~(\textbf{#1})} % Affiche le titre entre parenthèses et en gras s'il est fourni
  . % Point à la fin
}
{
  \end{mdframed}
}

% Définir un environnement pour encadrer les théorèmes avec un titre
\NewDocumentEnvironment{theorem}{O{}}
{
  \begin{mdframed}[linewidth=1pt,linecolor=darkgray,backgroundcolor=darkgray!10,roundcorner=5pt]
  \textbf{Théorème}%
  \IfNoValueTF{#1}{}{~(\textbf{#1})} % Affiche le titre entre parenthèses et en gras s'il est fourni
  . % Point à la fin
}
{
  \end{mdframed}
}


\theoremstyle{plain}
\newtheorem*{remark}{Remarque}
\newtheorem*{proposition}{Proposition}
\newtheorem*{lemma}{Lemme}
\newtheorem*{prop}{Propriété}
\newtheorem*{corollary}{Corollaire}
\newtheorem*{proof}{Démonstration}
\newtheorem*{example}{Exemple}


% OTHERS 


\newlength\tindent
\setlength{\tindent}{\parindent}
\setlength{\parindent}{0pt}
\renewcommand{\indent}{\hspace*{\tindent}}


\usepackage{fancyvrb}
\usepackage{tikz-cd} 
\usepackage{amsmath}
\usepackage{mathrsfs}  
\usepackage{amssymb}
\usepackage{tkz-graph}
\usepackage{caption}
\usepackage{multicol}
\usepackage{listings} % Importation du package listings
\usepackage{xcolor} % Pour ajouter de la couleur

\setlength{\columnsep}{1cm} % Espace entre les colonnes


% Définition du style de l'en-tête
\pagestyle{fancy}
\fancyhf{} % Nettoyer les en-têtes et pieds de page

% Gauche : Nom de la section
\fancyhead[L]{\nouppercase{\leftmark}}

% Droite : Logo
\fancyhead[R]{\includegraphics[width=2cm]{./images/logo_UT.png}}

% Ligne sous l'en-tête
\renewcommand{\headrulewidth}{0.4pt}

% Numéro de page centré en bas
\fancyfoot[C]{\thepage}

% Ligne au-dessus du pied de page (optionnel, mettre à 0pt pour supprimer)
\renewcommand{\footrulewidth}{0pt}


\setlength{\headheight}{1.5cm} % Ajuste la hauteur de l'en-tête



\definecolor{codegreen}{rgb}{0,0.6,0}
\definecolor{codegray}{rgb}{0.5,0.5,0.5}
\definecolor{codepurple}{rgb}{0.58,0,0.82}
\definecolor{backcolour}{rgb}{0.95,0.95,0.92}

\lstdefinestyle{mystyle}{
    backgroundcolor=\color{backcolour},   
    commentstyle=\color{codegreen},
    keywordstyle=\color{magenta},
    numberstyle=\tiny\color{codegray},
    stringstyle=\color{codepurple},
    basicstyle=\ttfamily\footnotesize,
    breakatwhitespace=false,         
    breaklines=true,                 
    captionpos=b,                    
    keepspaces=true,                 
    numbers=left,                    
    numbersep=5pt,                  
    showspaces=false,                
    showstringspaces=false,
    showtabs=false,                  
    tabsize=2
}

\lstset{
    style=mystyle
}