\documentclass[a4paper]{article}

% HEADER


\usepackage[top=3cm, bottom=3cm, left=3.5cm, right=3.5cm]{geometry}


\setlength{\parindent}{0cm}


\usepackage[french]{babel}
\usepackage[T1]{fontenc}
\usepackage{ragged2e}
\usepackage{amsfonts}
\usepackage{systeme}
\usepackage{amsmath}
\usepackage{amssymb}
\usepackage{comment}
\usepackage{multicol}
\usepackage{lipsum} 
\usepackage{graphicx}
\usepackage{stmaryrd}
\usepackage{systeme}
\usepackage{wrapfig}
\usepackage{colortbl}
\usepackage{cellspace}
\usepackage{stmaryrd}
\usepackage{ntheorem}
\usepackage{lmodern}
\usepackage{mathtools}
\usepackage{ragged2e}
\usepackage{tabularx}
\usepackage{titlepic}
\usepackage{fancyhdr}
\usepackage{caption}
\usepackage{xcolor} % pour les couleurs
\usepackage[linkbordercolor=white]{hyperref} % après avoir chargé xcolor
\usepackage{systeme}
\usepackage[T1]{fontenc}
\usepackage{lmodern}
\usepackage{listings}
\usepackage{tikz}
\usepackage{mdframed}
\usepackage{xparse} % Nécessaire pour définir des environnements avec arguments optionnels
\usepackage{booktabs}  % Pour des tableaux plus jolis
\usepackage{float} % For image positionning


% \setcounter{secnumdepth}{-1} % Désactive le compteur des parties

% PAGE SETTINGS

% \justifying

% \setlength{\columnseprule}{1pt}
% \def\columnseprulecolor{\color{black}}

% \newcolumntype{C}{>{$\displaystyle}Sc<$}
% \cellspacetoplimit=5pt
% \cellspacebottomlimit=5pt

\newcolumntype{C}{>{$\displaystyle}Sc<$}
\cellspacetoplimit=5pt
\cellspacebottomlimit=5pt


% MATHS SHORTHANDS

\newcommand{\C}{\mathbb{C}}
\newcommand{\R}{\mathbb{R}}
\newcommand{\Q}{\mathbb{Q}}
\newcommand{\Z}{\mathbb{Z}}
\newcommand{\N}{\mathbb{N}}
\newcommand{\U}{\mathbb{U}}
\newcommand{\K}{\mathbb{K}}
\newcommand{\M}{\mathcal{M}}
\newcommand{\B}{\mathcal{B}}

\renewcommand{\epsilon}{\varepsilon}
\renewcommand{\phi}{\varphi}
\renewcommand{\rho}{\varrho}

% MOD NOTATION

\theorembodyfont{\upshape}

% Définir un environnement pour encadrer les définitions avec un titre
\NewDocumentEnvironment{definition}{O{}}
{
  \begin{mdframed}[linewidth=0pt,linecolor=gray,backgroundcolor=gray!10,roundcorner=5pt]
  \textbf{Définition}%
  \IfNoValueTF{#1}{}{~(\textbf{#1})} % Affiche le titre entre parenthèses et en gras s'il est fourni
  . % Point à la fin
}
{
  \end{mdframed}
}

% Définir un environnement pour encadrer les théorèmes avec un titre
\NewDocumentEnvironment{theorem}{O{}}
{
  \begin{mdframed}[linewidth=1pt,linecolor=darkgray,backgroundcolor=darkgray!10,roundcorner=5pt]
  \textbf{Théorème}%
  \IfNoValueTF{#1}{}{~(\textbf{#1})} % Affiche le titre entre parenthèses et en gras s'il est fourni
  . % Point à la fin
}
{
  \end{mdframed}
}


\theoremstyle{plain}
\newtheorem*{remark}{Remarque}
\newtheorem*{proposition}{Proposition}
\newtheorem*{lemma}{Lemme}
\newtheorem*{prop}{Propriété}
\newtheorem*{corollary}{Corollaire}
\newtheorem*{proof}{Démonstration}
\newtheorem*{example}{Exemple}


% OTHERS 


\newlength\tindent
\setlength{\tindent}{\parindent}
\setlength{\parindent}{0pt}
\renewcommand{\indent}{\hspace*{\tindent}}


\usepackage{fancyvrb}
\usepackage{tikz-cd} 
\usepackage{amsmath}
\usepackage{mathrsfs}  
\usepackage{amssymb}
\usepackage{tkz-graph}
\usepackage{caption}
\usepackage{multicol}
\usepackage{listings} % Importation du package listings
\usepackage{xcolor} % Pour ajouter de la couleur

\setlength{\columnsep}{1cm} % Espace entre les colonnes


% Définition du style de l'en-tête
\pagestyle{fancy}
\fancyhf{} % Nettoyer les en-têtes et pieds de page

% Gauche : Nom de la section
\fancyhead[L]{\nouppercase{\leftmark}}

% Droite : Logo
\fancyhead[R]{\includegraphics[width=2cm]{./images/logo_UT.png}}

% Ligne sous l'en-tête
\renewcommand{\headrulewidth}{0.4pt}

% Numéro de page centré en bas
\fancyfoot[C]{\thepage}

% Ligne au-dessus du pied de page (optionnel, mettre à 0pt pour supprimer)
\renewcommand{\footrulewidth}{0pt}


\setlength{\headheight}{1.5cm} % Ajuste la hauteur de l'en-tête



\definecolor{codegreen}{rgb}{0,0.6,0}
\definecolor{codegray}{rgb}{0.5,0.5,0.5}
\definecolor{codepurple}{rgb}{0.58,0,0.82}
\definecolor{backcolour}{rgb}{0.95,0.95,0.92}

\lstdefinestyle{mystyle}{
    backgroundcolor=\color{backcolour},   
    commentstyle=\color{codegreen},
    keywordstyle=\color{magenta},
    numberstyle=\tiny\color{codegray},
    stringstyle=\color{codepurple},
    basicstyle=\ttfamily\footnotesize,
    breakatwhitespace=false,         
    breaklines=true,                 
    captionpos=b,                    
    keepspaces=true,                 
    numbers=left,                    
    numbersep=5pt,                  
    showspaces=false,                
    showstringspaces=false,
    showtabs=false,                  
    tabsize=2
}

\lstset{
    style=mystyle
}


\begin{document}

% ==================================================================================================================================
% TITLEPAGE 

\begin{titlepage}
    \begin{center}
        \includegraphics[width=0.4\textwidth]{./images/logo_UT.png} \\[1cm]

        {\Huge \textbf{Modèle de Réacteur Nucléaire}} \\[1cm]

        {\Large \textbf{UE : Simulation Aléatoire}} \\[2cm]

        \textbf{Auteurs :} \\[0.5cm]

        Marco Sanfilippo \\
        \href{mailto:marco.sanfilippo@univ-tlse3.fr}{marco.sanfilippo@univ-tlse3.fr}\\[0.5cm]

        Mathis Francine-Habas \\
        \href{mailto:mathis.francine-habas@univ-tlse3.fr}{mathis.francine-habas@univ-tlse3.fr}\\[0.5cm]

        Axel PIGEON \\
        \href{mailto:axel.pigeon@univ-tlse3.fr}{axel.pigeon@univ-tlse3.fr}\\[2cm]

    
        \textbf{Université :} Université de Toulouse \\[0.5cm]
        \textbf{Date :} \today \\[2cm]

        \vfill

        % {\large Rapport dans le cadre d'un projet de développement de modèles d'intelligence 
        % artificielle pour déterminer la présence d'un molécule dans un environnement à partir 
        % du déplacement d'une planaire. }
    \end{center}
\end{titlepage}


\tableofcontents

\noindent\rule{\textwidth}{1pt}

% ==================================================================================================================================
% Introduction 

\section*{Introduction}




\newpage 

% ==================================================================================================================================
% La fission nucléaire

\section{La fission nucléaire}

\subsection{Un peu de chimie}

Avant de modéliser un réacteur nucléaire, commençons par nous pencher sur ses mécanismes internes pour comprendre 
leur fonctionnement. 

En physique, il existe plusieurs réactions dites chimiques (évaporation, solidification, transcendance, etc...) que nous 
connaissons depuis la maternelle. Ces réactions font partie de la vie courante et sont de fait, très intuitives. 
Elles altèrent et modifient les propriétés physiques macroscopiques des matériaux : il est facile de distinguer de l'eau 
liquide de l'eau solide. 
Si l'on regarde du côté microscopique, elles modifient directement l'agencement des molécules dans l'espace. 
À l'état gazeux, les molécules bougent de manière frénétique dans l'espace sans structure apparente. À l'état liquide, elles 
se présentent sous une forme plus compacte comparable à un tas de sable où les grains représentent les molécules. 
Enfin, l'état solide se caractérise par un agencement très précis des molécules en une structure géométrique.

\begin{figure}[h]
    \centering
    \includegraphics[scale=0.4]{images/etats_matiere.png}
    \caption{États de la matière}
\end{figure}

Cependant, certaines transformations mettent en jeu non plus les molécules, mais les noyaux des atomes eux-mêmes. 
Ces réactions, dites nucléaires, libèrent des quantités d’énergie bien supérieures aux réactions chimiques.

\subsection{La fission} 

Il existe deux types de réactions nucléaires : la \emph{fission} et la \emph{fusion}. 
La {fission nucléaire} fut découverte en 1938 par des physiciens allemands sur le travail de Enrico Fermi datant des années 
1934. Cette découverte fait l'effet d'une bombe (sans mauvais jeu de mots) dans le milieu de la physique moderne et lance 
aussitôt la course au développement de l'arme atomique dans un contexte tendu de fin seconde guerre mondiale et 
de début de geurre froide. 

Le principe de la fission nucléaire est assez simple. Partant d'un noyau lourd instable (souvent une isotope 
de l'uranium ou du plutonium) et en le bombardant de neutrons à très grande vitesse on souhaite séparer le 
noyau en deux parties approximativement égales d'éléments moins lourds. 
Lors de cette séparation, des neutrons sont libérés et peuvent, eux aussi, produire de nouvelles réactions de fission 
avec d'autres atomes lourds. C'est ce que l'on appelle le principe de \emph{réaction en chaîne}. 

Une telle réaction produit énormément d'énergie et de chaleur, à titre d'exemple, la combustion de 1kg de charbon
produit $3 \times 10^7$ J alors la fission de 1kg d'uranium en produit $ 8 \times 10^{13}$ soit entre 
100 millions et 1 milliard de fois plus d'énergie libérée. 

Ce surplus d'énergie permet actuellement de produire de l'électricité dans nos centrales nucléaires. La chaleur 
produite vaporise de l'eau qui fait ensuite tourner des turbines pour produire de l'électricité. 

\subsection{Différents types de neutrons}

Pour induire une réaction en chaîne dans un réacteur et la maintenir, il faut jouer avec différents paramètres tels que 
la densité du matériaux fissile, la quantité de neutrons introduite au départ, la forme du réacteur, sa température, etc... 
Nous essaierons d'introduire tout ces paramètres dans notre modèle pour le rendre le plus proche du réel possible. 

Dans cet objectif, nous ne pouvons pas négliger le fait que plusieurs types de neutrons existent dans ce type de réactions. 
En effet, ceux-ci possèdent une certaine quantité d'énergie et se déplacent à une certaine vitesse qui n'est pas toujours constante. 
Ainsi, pour pouvoir induire une réaction de fission, un neutron doit être suffisament ralenti pour percuter un atome lourd. 
Nous différencierons donc 3 types de neutrons : 
\begin{itemize}
    \item Les neutrons \textbf{rapides} : produits directement par une réaction de fission, ils possèdent beaucoup 
    d'énergie, se déplacent vite et ont une très faible probabilité d'être absorbés par le milieu. 
    \item Les neutrons \textbf{épithermiques} : ce sont des neutrons en ralentissement dans le milieu possédant une 
    énergie plus faible. Ils ont plus d'interractions avec le milieu. 
    \item Les neutrons \textbf{thermiques} : ce sont des neutrons complètement ralentis qui se déplacent à la même vitesse 
    que les isotopes lourds. Ils peuvent donc plus facilement induire des réactions de fission. Ils servent 
    à maintenir une réaction en chaîne dans le réacteur. 
\end{itemize}

\begin{table}[H]
    \centering
    \begin{adjustbox}{max width=\textwidth}
        \begin{tabular}{@{}>{\RaggedRight\arraybackslash}p{3cm}
                            >{\RaggedRight\arraybackslash}p{3.5cm}
                            >{\RaggedRight\arraybackslash}p{4cm}
                            >{\RaggedRight\arraybackslash}p{4cm}@{}}
            \toprule
            \textbf{Type de neutron} & \textbf{Plage d'énergie typique} & \textbf{Rôle dans le réacteur} & \textbf{Comportement} \\ 
            \midrule
            \textbf{Rapide} & $\sim$2\,MeV & Produit directement par la fission & Subit des collisions pour se ralentir (modération) \\
            \textbf{Épithermique} & 1\,eV – 1\,keV & Phase intermédiaire du ralentissement & Perd progressivement son énergie cinétique \\
            \textbf{Thermique} & $\sim$0.025\,eV & Provoque efficacement la fission de l'U-235 & Absorbé par les noyaux fissiles \\ 
            \bottomrule
        \end{tabular}
    \end{adjustbox}
    \caption{Principaux types de neutrons intervenant dans un réacteur nucléaire}
    \label{tab:types_neutrons}
\end{table}


\subsection{Le rôle du modérateur}

On souhaiterai entretenir une réaction en chaîne dans un réacteur tout en évitant qu'elle s'emballe, nous donc devons 
introduire un \emph{modérateur}. 
Son rôle sera de ralentir les neutrons nouvellement créés par fission tout en en absorbant une certaine quantité pour ne 
pas que la réaction s'emballe. Un modérateur doit donc être assez léger tout en ralentissant suffisament les neutrons. 
En pratique, différents matériaux sont utilisés, les voici dans le tableau suivant : 

\begin{table}[H]
    \centering
    \begin{adjustbox}{max width=\textwidth}
        \begin{tabular}{@{}>{\RaggedRight\arraybackslash}p{3.2cm}
                            >{\RaggedRight\arraybackslash}p{3.2cm}
                            >{\RaggedRight\arraybackslash}p{3cm}
                            >{\RaggedRight\arraybackslash}p{3cm}
                            >{\RaggedRight\arraybackslash}p{3cm}@{}}
            \toprule
            \textbf{Modérateur} & \textbf{Composition chimique} & \textbf{Pouvoir modérateur} & \textbf{Absorption neutronique} & \textbf{Utilisation typique} \\ 
            \midrule
            \textbf{Eau légère (H$_2$O)} & Hydrogène et oxygène & Excellent & Moyenne & Réacteurs à eau pressurisée (REP) \\
            \textbf{Eau lourde (D$_2$O)} & Deutérium et oxygène & Excellent & Très faible & Réacteurs CANDU (Canada) \\
            \textbf{Graphite (C)} & Carbone pur & Bon & Très faible & Réacteurs RBMK et AGR \\
            \textbf{Béryllium (Be)} & Béryllium pur & Bon & Faible & Réacteurs de recherche \\ 
            \bottomrule
        \end{tabular}
    \end{adjustbox}
    \caption{Caractéristiques comparées des principaux matériaux modérateurs}
    \label{tab:moderateurs}
\end{table}

\subsection{La section efficace}

Pour modéliser correctement les interactions entre les neutrons et la matière, il est nécessaire d’introduire la notion de 
\emph{section efficace}. Celle-ci traduit la probabilité qu’un neutron interagisse avec un noyau lorsqu’il le rencontre. 
En d'autres termes, elle représente une surface fictive caractérisant la capacité d’un noyau à provoquer une réaction 
(nucléaire ou non) lorsqu’il est bombardé par des neutrons.

La section efficace, notée $\sigma$, s’exprime en \emph{barn} :
    \[ 1\,\text{barn} = 10^{-28}\,\text{m}^2 \]
Plus la valeur de $\sigma$ est grande, plus la probabilité d’interaction est élevée. Elle dépend à la fois :
\begin{itemize}
    \item de la nature du noyau cible (Uranium, Plutonium, etc.),
    \item du type d’interaction considérée (diffusion, capture, fission),
    \item et surtout de l’énergie du neutron incident.
\end{itemize}

La figure suivante illustre typiquement la variation de la section efficace de fission en fonction de l’énergie du neutron : 
elle est très faible pour les neutrons rapides, et augmente considérablement lorsque les neutrons deviennent thermiques. 
C’est cette dépendance énergétique qui rend le rôle du modérateur essentiel dans les réacteurs à neutrons thermiques.

\begin{figure}[H]
    \centering
    \includegraphics[scale=0.25]{images/section_efficace.png}
    \caption{Évolution typique de la section efficace de fission en fonction de l’énergie du neutron}
    \label{fig:section_efficace}
\end{figure}

On distingue généralement plusieurs types de sections efficaces :
\begin{itemize}
    \item $\sigma_f$ : \textbf{section efficace de fission}, probabilité qu’un neutron provoque une fission,
    \item $\sigma_c$ : \textbf{section efficace de capture}, probabilité qu’un neutron soit absorbé sans fission,
    \item $\sigma_s$ : \textbf{section efficace de diffusion}, probabilité qu’un neutron soit simplement dévié.
\end{itemize}

Leur somme constitue la \emph{section efficace totale} :
    \[ \sigma_t = \sigma_f + \sigma_c + \sigma_s \]

Pour illustrer ces valeurs, le tableau suivant compare les sections efficaces de différents isotopes couramment utilisés 
dans les réacteurs nucléaires pour des neutrons thermiques :

\begin{table}[H]
    \centering
    \begin{adjustbox}{max width=\textwidth}
        \begin{tabular}{@{}>{\RaggedRight\arraybackslash}p{3cm}
                            >{\RaggedRight\arraybackslash}p{3.5cm}
                            >{\RaggedRight\arraybackslash}p{3.5cm}
                            >{\RaggedRight\arraybackslash}p{3.5cm}@{}}
            \toprule
            \textbf{Isotope} & \textbf{Section efficace de fission ($\sigma_f$)} & \textbf{Section efficace de capture ($\sigma_c$)} & \textbf{Remarques} \\ 
            \midrule
            \textbf{Uranium-235} & $\sim$580 barns & $\sim$100 barns & Très réactif aux neutrons thermiques \\ 
            \textbf{Uranium-238} & $\sim$0.02 barns & $\sim$2.7 barns & Ne fissionne qu’avec des neutrons rapides \\ 
            \textbf{Plutonium-239} & $\sim$750 barns & $\sim$270 barns & Fortement fissile, produit secondaire du cycle U-238 \\ 
            \bottomrule
        \end{tabular}
    \end{adjustbox}
    \caption{Comparaison des sections efficaces pour différents isotopes fissiles}
    \label{tab:sections_efficaces}
\end{table}

Ainsi, la grande différence de section efficace entre l’uranium-235 et l’uranium-238 explique pourquoi seul le premier 
est utilisé comme combustible principal dans les réacteurs à neutrons thermiques. 
La compréhension et la maîtrise de cette grandeur sont donc essentielles pour modéliser la probabilité d’absorption, 
de diffusion et de fission dans un réacteur.


% ==================================================================================================================================
% Modélisation 

\section{Modélisation}



% ==================================================================================================================================
% Études statistiques 

\section{Études Statistiques}





% Sources 


% https://fr.wikipedia.org/wiki/Fission_nucl%C3%A9aire


\end{document}